% Diese Zeile bitte -nicht- aendern.
\documentclass[course=erap]{aspdoc}

%%%%%%%%%%%%%%%%%%%%%%%%%%%%%%%%%
%% TODO: Ersetzen Sie in den folgenden Zeilen die entsprechenden -Texte-
%% mit den richtigen Werten.
\newcommand{\theGroup}{213} % Beispiel: 42
\newcommand{\theNumber}{A326} % Beispiel: A123
\author{Noah Schlenker \and Leon Baptist Kniffki \and Christian Krinitsin}
\date{Sommersemester 2023} % Beispiel: Wintersemester 2019/20
%%%%%%%%%%%%%%%%%%%%%%%%%%%%%%%%%

% Diese Zeile bitte -nicht- aendern.
\title{Gruppe \theGroup{} -- Abgabe zu Aufgabe \theNumber}

\begin{document}
\maketitle

\section{Einleitung}

Verwendung von Wurzel 2:
\begin{itemize}
  \item Das verhältnis der beiden seitenlängen eines blattes im din-a-format beträgt 1 / sqrt(2) mit rundung auf 
        ganze millimeter. Dadurch ist sichergestellt, dass bei halbierung des blattes entlang der längeren seite wieder 
        ein blatt im din-a-format (mit um eins erhöhter nummerierung) entsteht.
  \item Die wurzel aus 2 ist das frequenzverhältnis zweier töne in der musik bei gleichschwebender stimmung, die einen tritonus, 
        also eine halbe oktave bilden.
  \item In der elektrotechnik enthält die beziehung zwischen scheitelwert und effektivwert von sinusförmiger wechselspannung ebenfalls die konstante 
        sqrt(2).
\end{itemize}

Geschichte über Näherung der Wurzel:
\begin{itemize}
  \item Die alten Inder schätzen $\sqrt{2} \approx \tfrac {577}{408} = 1,414215686$… . Stimmt auf 5 Nachkommastellen. Abweichung beträgt nur +0,0001502 Prozent.
  \item Babylonier aus 1800 v. Chr.: $\tfrac {30547}{21600} = 1,414212962$… . Abweichung von -0,0000424 Prozent.
\end{itemize}

Möglichkeit, $\sqrt{2}$ mit einer unendlichen Präzision zu berechnen. Vorwegnahme: Bignums, Newton-Raphson-Division, Karazuba, Matrixexponentiation. \\
(0,75 Seiten)

\section{Lösungsansatz}

\subsection{Big-Num}
Diskussion über die Notwendigkeit von Bignums, Erklärung der Implementierung (Little Endian, Arithmetische Operationen, nicht Division). Laufzeiten thematisieren? \\
(1 Seite)

\subsection{Karazuba-Multiplikation}
Erklärung des Algorithmus und der Umsetzung im Code. \\
(0,75 Seiten)

\subsection{Newton-Raphson-Division}
Erklärung des Algorithmus und der Umsetzung im Code. \\
(1 Seite)

% TODO: Je nach Aufgabenstellung einen der Begriffe wählen
\section{Korrektheit/Genauigkeit}
Wahrscheinlich Genauigkeit, da es die Aufgabe ist, $\sqrt{2}$ beliebig genau darzustellen. \\

Umfangreiche Erklärung darüber, wie die Matrix Elemente an $\sqrt{2}$ konvergiert und Newton-Raphson and die Division. Erklärung, wie die Kombination aus Bignum und Fixkommazahlen unendliche Genauigkeit 
ermöglicht, auf Kosten von Laufzeit, die im nächsten Kapitel beleuchtet wird. \\ 
(1,5 - 2 Seiten?)

\section{Performanzanalyse}
Newton Raphson Laufzeit erklären, mit Graphiken demonstrieren, das selbe mit Exponentiation. \\
Vergleichsimplementierungen ansetzen (SIMD, nicht SIMD? / Karazuba, normale Multiplikation / Bitshifts?), Graphisch laufzeiten vergleichen, tatsächliche Performanz erklären und schlussfolgern. \\
(2 - 3 Seiten)

\section{Zusammenfassung und Ausblick}
Ziel erreicht, unendliche Präzision ist gegeben. Nutzer können, abhängig von ihren Anforderungen oder "Computerspezifikationen", die Wurzel von 2 mit diesem Programm berechnen. \\
Ausblick: SIMD in AVX, mithilfe von 256 Bit kann man Multiplikationen noch schneller machen. Division lässt sich wahrscheinlich nicht optimieren, da SIMD nicht verwendet werden kann, 
und immer eine gewisse Anzahl an Iterationen gebraucht wird. \\
(0,75 - 1 Seite) \\
(Insgesamt 6 - 7,75 Seiten, 2 - 4 Seiten fehlen!)

% TODO: Fuegen Sie Ihre Quellen der Datei Ausarbeitung.bib hinzu
% Referenzieren Sie diese dann mit \cite{}.
% Beispiel: CR2 ist ein Register der x86-Architektur~\cite{intel2017man}.
\bibliographystyle{plain}
\bibliography{Ausarbeitung}{}

\end{document}
